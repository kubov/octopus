\section{Podsumowanie}
Praca zespołowa zdecydowanie przyczyniła się do szybkiego zaimplementowania projektu. Wspólna praca nie tylko pozwoliła na wymianę doświadczeń, ale również umożliwiła nam dopracowanie warsztatu. Podczas pracy zastosowano popularną w firmach programistycznych metodę polegającą na uprzednim sprawdzeniu kodu przez innego programistę przed włączeniem go do gałęzi (\emph{code review}). Dzięki temu, popełniane przez nas błędy zostawały wskazywane po to, aby nie popełniać ich w przyszłości.

Podczas prac udało się zrealizować wszystkie cele, jednak system nadal można usprawnić o polepszoną grafikę, udźwiękowianie oraz poprawioną wydajność.

Podsumowując, praca zespołowa przyczynia się nie tylko do przyspieszenia procesu wytwarzania oprogramowania, ale również pozwala na lepszy i automatyczny rozwój umiejętności współpracowników.
\pagebreak

\begin{thebibliography}{9}
\bibitem{PCL}
  Peter Seibel
  \emph{Practical Common Lisp}
  Apress,
  2009.

  \bibitem{DIH}
  Mark Pilgrim
  \emph{Dive into HTML5}
  http://diveintohtml5.info/

  \bibitem{JQ}

  \emph{jQuery Fundamentals}
  http://jqfundamentals.com/chapter/javascript-basics
  Bocoup.

  \bibitem{SBCL}
  \emph{Steel Bank Common Lisp Manual}
  http://www.sbcl.org/manual/index.html

\end{thebibliography}


